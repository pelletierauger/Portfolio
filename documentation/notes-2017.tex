%!TEX root = structure.tex

\newpage
\section{Nouvelle refonte du site Web}

Il y a beaucoup de nouvelles choses à accomplir pour faire de mon site Web un endroit grâce auquel je pourrais partager mon travail de manière optimale.

\begin{enumerate}  
\item Refaire la page d'ouverture, la présentation des projets principaux, en utilisant le nouveau design réalisé récemment.
\item Écrire le contenu des projets.
\item Le plus gros casse-tête : pouvoir inclure des sketchs \textit{p5.js} dans les pages de mes projets et dans les messages de mon blog.
\item Pouvoir inclure des mathématiques \LaTeX\ avec \textit{MathJax-node} dans les messages de mon blog.
\item Pouvoir inclure du code formatté avec \textit{Highlight.js} dans les messages de mon blog.
\item Écrire une meilleure présentation dans la page \textit{À propos}.
\item Créer une section \textit{Archive} qui présenterait aussi une mappemonde de mes projets, une représentation visuelle de tous les projets que j'ai créés et des connexions entre eux. Il pourrait peut-être même y avoir une place pour les projets non réalisés, pour les \textit{rêves}. Je pourrais définir chacun de mes projets selon certains mots-clés, et selon certaines relations qu'il entretient avec les autres projets, et créer un système où les projets s'organiseraient spatialement eux-mêmes. Ensuite, je dessinerais la position de chaque projet avec une ellipse, et ses connexions avec les autres projets par des lignes. La page \textit{Archive} inclurait la mappemonde, une auto-documentation expliquant comment cette mappemonde a été réalisée, et finalement, au bas de la page, une liste visuellement très sobre de tous les projets. En plaçant le curseur au-dessus d'un point, on obtiendrait un \textit{DIV} superposé à la mappemonde, qui présenterait sommairement le projet et donnerait un hyperlien. Les points devraient décider eux-mêmes de quelles couleurs ils veulent colorer leur propre partie de la mappemonde. Ainsi, la mappemonde pourrait changer complètement à chaque nouveau projet que j'y ajouterais. La visibilité des projets non réalisés pourrait être optionnelle. Les projets pourraient avoir des points plus petits ou plus gros selon la quantité de lignes de code qu'ils contiennent. Et une masse plus grande, aussi! Si je calcule la masse dans mon système de particules, les gros points massifs se déplaceront plus lentement que les petits points agiles et légers. Ce sera beau.
\end{enumerate}

\section{Intégration de sketchs \textit{p5.js} sur le site Web}

C'est un bon casse-tête. Le pire élément du casse-tête est la nécessité de pouvoir inclure des sketchs \textit{p5.js} dans des notes de blogs, donc possiblement plusieurs sketchs dans le même fichier HTML. Évidemment que, conséquamment, tous les sketchs \textit{p5.js} que je vais inclure sur mon site Web devront être programmés en \textit{instance mode}. Ce n'est pas un très gros problème en soi. Le plus gros problème est de créer un environnement dans lequel je peux créer aisément mes sketchs et ensuite les déployer aussi aisément.