%!TEX root = structure.tex

\begin{abstract}
Ce portfolio d'art numérique est un site statique généré avec Node.js.
\end{abstract}

\section{Génération d'un site Web statique avec Node.js}
Je crée un portfolio en ligne pour mon travail en art numérique et je planifie le mettre à jour fréquemment. Le processus de mise à jour doit être aussi simple que possible. Mon but est de généré le contenu HTML en utilisant une série de modules Node.js. Chacun de ces modules sera traité en utilisant le module \textit{fs} (pour \textit{file system}) de Node.js.

Voici ce que j'ai en ce moment :

\begin{lstlisting}
var fs = require("fs");

var header = `<!DOCTYPE html>
<html>
<head>
    <meta charset="UTF-8">
    <title>Guillaume Pelletier-Auger</title>
</head>
`;

var body = `
<body><h1>I'm writing my own HTML files with Node.js. It's pretty neat.</h1>
It's actually like... awesome! I can now think of a website as a data structure.
</body>
</html>
`;

fs.writeFile('index2.html', header + body, function(err) {
    if (err) {
        return console.error(err);
    } else {
        console.log("Data written successfully!");
    }
});

\end{lstlisting}

Je songeais précédemment à utiliser un générateur de sites Web statiques (\textit{Static Site Engine}) comme Harp.js, mais il m'est clair maintenant que ce sera beaucoup plus simple de tout faire moi-même dans Node.js. Mon système aura ainsi une bien plus grande indépendance fonctionnelle.

\section{Un site Web est une structure de données}
Je dois penser à mon portfolio comme étant une structure de données.
J'ai commencé avec ce fichier JSON :
\begin{lstlisting}
{
    header: "<head></head><body><h1></h1>",
    items: {
        oscillators: {
            title_fr: "Les Oscillateurs",
            title_en: "Oscillators",
            content_fr: `
            
            `,
            content_en: `
            `,
        },
        joy_and_confusion: {

        },
        dunes: {
        }
    }
}
\end{lstlisting}

Je dois avoir une fonction qui ressemble à ça :

\begin{lstlisting}
function buildWorks() {
    var content = ``;
    for (var i = 0; i < portfolio.items.length; i++)  {
        var title = portfolio.items[i].title;
        content = content + `<img src="${title}.jpg"></a>`;
    }
}
\end{lstlisting}
\clearpage
\section{Liste de fichiers comme base de données}
Mon système sera beaucoup plus efficace si chaque page de mon portfolio correspond à un fichier différent dans ma base de données. Il me serait en effet désagréable de travailler sur les diverses pages de mon portfolio si elles étaient toutes écrites dans le même fichier JavaScript ou JSON. Ça deviendrait rapidement lourd et pas clair.

En utilisant la méthode \textit{readdirSync} du module \textit{fs} de Node.js, j'obtiens la liste complète des fichiers qui sont dans le dossier \textit{Pages}. Les fichiers qui seront dans le dossier \textit{Pages} seront des modules Node.js (d'où l'utilisation de la méthode \textit{require}) qui renvoient un objet JavaScript littéral. L'objet contient les propriétés fr, en et link.

\begin{lstlisting}
var fs = require("fs");
var files = fs.readdirSync('pages');
var page = require('./pages/' + files[0]);
\end{lstlisting}

Voici à quoi ressemble une page définie comme module Node.js, pour l'instant :
\begin{lstlisting}
exports.fr = {
    title: "Les Dunes",
    description: "Une série d'image générées par des fonctions itératives.",
    content: `<i>Les Dunes</i> est une série d'images générées par une variété de fonctions itératives.
    Elles ont été créées avec p5.js.`
};

exports.en = {
    title: "Dunes",
    description: "A series of images generated by iterated functions.",
    content: `<i>Dunes</i> is a series of images generated by various iterated functions. 
    They were created in JavaScript with the p5.js library.`
}

exports.link = null;
\end{lstlisting}

La fonction qui va générer toutes mes pages doit donc traiter ces données. Elle doit :
\begin{enumerate}  
\item Vérifier que toutes les données sont présentes.
\item Vérifier si la propriété \textit{link} est \textit{null} ou non. 
\begin{enumerate} 
\item Si \textit{link} est \textit{null}, ça veut dire qu'il s'agit d'une page qui sera hébergée directement sur mon site Web. La fonction doit donc créer un fichier HTML en utilisant les données contenues dans l'objet renvoyé par le module.
\item Si la propriété n'est pas \textit{null}, ça veut dire que cet élément du portfolio est hébergé sur un site externe. La fonction ne crée donc aucun fichier HTML pour cet élément du portfolio.
\end{enumerate} 
\item La fonction doit utiliser chaque élément dans le dossier \textit{Pages} pour générer les fichiers index.html, index-en.html et index-fr.html. Forcément, la fonction doit aussi lire dans un autre dossier, nommé \textit{Thumbnails}, qui contiendra une image pour chaque élément du portfolio.

% \item The third etc \ldots 
\end{enumerate}

Ma boucle pour traiter chacun des fichiers dans le dossier \textit{pages} ressemblera donc à ça :

\begin{lstlisting}
var fs = require("fs");
var files = fs.readdirSync('pages');
for (var i = 0; i < files.length; i++) {
    var page = require('./pages/' + files[i]);
    //Do stuff here.
}
generateMosaic(files, "fr");
generateMosaic(files, "en");

function generateMosaic(files, language) {
    for (var i = 0; i < files.length; i++) {
        var page = require('./pages/' + files[i]);

        if (language == "fr") {
            var pageLang = page.fr;
        } else if (language == "en") {
            var pageLang = page.en;
        }
        
        var title = pageLang.title;
        var description = pageLang.description;
    
        //I must reformat the filename using a regular expression. 
        //Here is the RegEx that I need : .*\.\b
        //I must also find a way to consider the language choice.
        var link = (page.link || "./" + files[i]);
    }
}
\end{lstlisting}



\section{Création des liens de navigation}
La création des liens de navigation sera clairement la partie la plus complexe de mon système.